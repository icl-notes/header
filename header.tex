\documentclass{article}

% Packages and formatting

% Packages
\usepackage{amssymb}
\usepackage{amsthm}
\usepackage[UKenglish]{babel}
\usepackage{commath}
\usepackage{enumitem}
\usepackage{fancyhdr}
\usepackage[margin=1in]{geometry}
\usepackage{graphicx}
\usepackage[utf8]{inputenc}
\usepackage{hyperref}
\usepackage{listings}
\usepackage{mathtools}
\usepackage{stmaryrd}
\usepackage{tikz-cd}
\usepackage{csquotes}

% Formatting
\addto\captionsUKenglish{\renewcommand{\abstractname}{Syllabus}}
\delimitershortfall5pt
\ifx\thm\undefined\newtheorem{n}{}\else\newtheorem{n}{}[\thm]\fi
\let\g\gg
\let\l\ll
\setlength{\parindent}{0cm}

% Environments

% Plain environments
\theoremstyle{plain}
\newtheorem{algorithm}[n]{Algorithm}
\newtheorem*{algorithm*}{Algorithm}
\newtheorem{conjecture}[n]{Conjecture}
\newtheorem*{conjecture*}{Conjecture}
\newtheorem{corollary}[n]{Corollary}
\newtheorem*{corollary*}{Corollary}
\newtheorem{lemma}[n]{Lemma}
\newtheorem*{lemma*}{Lemma}
\newtheorem{proposition}[n]{Proposition}
\newtheorem*{proposition*}{Proposition}
\newtheorem{theorem}[n]{Theorem}
\newtheorem*{theorem*}{Theorem}

% Definition environments
\theoremstyle{definition}
\newtheorem{aim}[n]{Aim}
\newtheorem*{aim*}{Aim}
\newtheorem{axiom}[n]{Axiom}
\newtheorem*{axiom*}{Axiom}
\newtheorem{condition}[n]{Condition}
\newtheorem*{condition*}{Condition}
\newtheorem{definition}[n]{Definition}
\newtheorem*{definition*}{Definition}
\newtheorem{example}[n]{Example}
\newtheorem*{example*}{Example}
\newtheorem{exercise}[n]{Exercise}
\newtheorem*{exercise*}{Exercise}
\newtheorem{fact}[n]{Fact}
\newtheorem*{fact*}{Fact}
\newtheorem{goal}[n]{Goal}
\newtheorem*{goal*}{Goal}
\newtheorem{law}[n]{Law}
\newtheorem*{law*}{Law}
\newtheorem{plan}[n]{Plan}
\newtheorem*{plan*}{Plan}
\newtheorem{problem}[n]{Problem}
\newtheorem*{problem*}{Problem}
\newtheorem{question}[n]{Question}
\newtheorem*{question*}{Question}
\newtheorem{warning}[n]{Warning}
\newtheorem*{warning*}{Warning}

% Remark environments
\theoremstyle{remark}
\newtheorem{acknowledgements}[n]{Acknowledgements}
\newtheorem*{acknowledgements*}{Acknowledgements}
\newtheorem{annotations}[n]{Annotations}
\newtheorem*{annotations*}{Annotations}
\newtheorem{assumption}[n]{Assumption}
\newtheorem*{assumption*}{Assumption}
\newtheorem{conclusion}[n]{Conclusion}
\newtheorem*{conclusion*}{Conclusion}
\newtheorem{claim}[n]{Claim}
\newtheorem*{claim*}{Claim}
\newtheorem{notation}[n]{Notation}
\newtheorem*{notation*}{Notation}
\newtheorem{note}[n]{Note}
\newtheorem*{note*}{Note}
\newtheorem{remark}[n]{Remark}
\newtheorem*{remark*}{Remark}

% General commands

% Lecture
\newcommand{\lecture}[3]{
  \marginpar{
    Lecture #1 \\
    #2 \\
    #3
  }
}

% Brackets
\renewcommand{\eval}[1]{\left. #1 \right|}          % Evaluation
\newcommand{\br}{\del}                              % Brackets
\newcommand{\abr}[1]{\left\langle #1 \right\rangle} % Angle brackets
\newcommand{\fbr}[1]{\left\lfloor #1 \right\rfloor} % Floor brackets
\newcommand{\lbr}[1]{\left\lfloor #1 \right\rfloor} % Ceiling brackets

% Function
\newcommand{\function}[5][]{
  \ifx &#1&
    \begin{array}{rcl}
      #2 & \longrightarrow & #3 \\
      #4 & \longmapsto     & #5
    \end{array}
  \else
    \fullfunction{#1}{#2}{#3}{#4}{#5}
  \fi
}

% Spaces
\newcommand{\F}{\mathbb{F}}   % Finite fields
\newcommand{\N}{\mathbb{N}}   % Natural numbers
\newcommand{\Z}{\mathbb{Z}}   % Integral numbers
\newcommand{\Q}{\mathbb{Q}}   % Rational numbers
\newcommand{\R}{\mathbb{R}}   % Real numbers
\newcommand{\C}{\mathbb{C}}   % Complex numbers
\renewcommand{\H}{\mathbb{H}} % Quaternion numbers
\newcommand{\A}{\mathbb{A}}   % Affine spaces
\renewcommand{\P}{\mathbb{P}} % Projective spaces

% Correspondence
\newcommand{\correspondence}[2]{
  \cbr{
    \begin{array}{c}
      #1
    \end{array}
  }
  \qquad
  \leftrightsquigarrow
  \qquad
  \cbr{
    \begin{array}{c}
      #2
    \end{array}
  }
}

% Integrals
\newcommand{\intd}[4]{\int_{#1}^{#2} \, #3 \, \dif #4}                      % Single integral
\newcommand{\iintd}[4]{\iint_{#1} \, #2 \, \dif #3 \, \dif #4}              % Double integral
\newcommand{\iiintd}[5]{\iiint_{#1} \, #2 \, \dif #3 \, \dif #4 \, \dif #5} % Triple integral

% Matrices
\newcommand{\onebytwo}[2]{     % One by two matrix
  \begin{pmatrix}
    #1 & #2
  \end{pmatrix}
}
\newcommand{\onebythree}[3]{   % One by three matrix
  \begin{pmatrix}
    #1 & #2 & #3
  \end{pmatrix}
}
\newcommand{\twobyone}[2]{     % Two by one matrix
  \begin{pmatrix}
    #1 \\
    #2
  \end{pmatrix}
}
\newcommand{\twobytwo}[4]{     % Two by two matrix
  \begin{pmatrix}
    #1 & #2 \\
    #3 & #4
  \end{pmatrix}
}
\newcommand{\threebyone}[3]{   % Three by one matrix
  \begin{pmatrix}
    #1 \\
    #2 \\
    #3
  \end{pmatrix}
}
\newcommand{\threebythree}[9]{ % Three by three matrix
  \begin{pmatrix}
    #1 & #2 & #3 \\
    #4 & #5 & #6 \\
    #7 & #8 & #9
  \end{pmatrix}
}


% Frakturs
\renewcommand{\aa}{\mathfrak{a}} % Fraktur a
\newcommand{\bb}{\mathfrak{b}}   % Fraktur b
\newcommand{\cc}{\mathfrak{c}}   % Fraktur c
\newcommand{\dd}{\mathfrak{d}}   % Fraktur d
\newcommand{\ee}{\mathfrak{e}}   % Fraktur e
\newcommand{\ff}{\mathfrak{f}}   % Fraktur f
\renewcommand{\gg}{\mathfrak{g}} % Fraktur g
\newcommand{\hh}{\mathfrak{h}}   % Fraktur h
\newcommand{\ii}{\mathfrak{i}}   % Fraktur i
\newcommand{\jj}{\mathfrak{j}}   % Fraktur j
\newcommand{\kk}{\mathfrak{k}}   % Fraktur k
\renewcommand{\ll}{\mathfrak{l}} % Fraktur l
\newcommand{\mm}{\mathfrak{m}}   % Fraktur m
\newcommand{\nn}{\mathfrak{n}}   % Fraktur n
\newcommand{\oo}{\mathfrak{o}}   % Fraktur o
\newcommand{\pp}{\mathfrak{p}}   % Fraktur p
\newcommand{\qq}{\mathfrak{q}}   % Fraktur q
\newcommand{\rr}{\mathfrak{r}}   % Fraktur r
\renewcommand{\ss}{\mathfrak{s}} % Fraktur s
\renewcommand{\tt}{\mathfrak{t}} % Fraktur t
\newcommand{\uu}{\mathfrak{u}}   % Fraktur u
\newcommand{\vv}{\mathfrak{v}}   % Fraktur v
\newcommand{\ww}{\mathfrak{w}}   % Fraktur w
\newcommand{\xx}{\mathfrak{x}}   % Fraktur x
\newcommand{\yy}{\mathfrak{y}}   % Fraktur y
\newcommand{\zz}{\mathfrak{z}}   % Fraktur z

% Calligraphics
\renewcommand{\AA}{\mathcal{A}} % Calligraphic A
\newcommand{\BB}{\mathcal{B}}   % Calligraphic B
\newcommand{\CC}{\mathcal{C}}   % Calligraphic C
\newcommand{\DD}{\mathcal{D}}   % Calligraphic D
\newcommand{\EE}{\mathcal{E}}   % Calligraphic E
\newcommand{\FF}{\mathcal{F}}   % Calligraphic F
\newcommand{\GG}{\mathcal{G}}   % Calligraphic G
\newcommand{\HH}{\mathcal{H}}   % Calligraphic H
\newcommand{\II}{\mathcal{I}}   % Calligraphic I
\newcommand{\JJ}{\mathcal{J}}   % Calligraphic J
\newcommand{\KK}{\mathcal{K}}   % Calligraphic K
\newcommand{\LL}{\mathcal{L}}   % Calligraphic L
\newcommand{\MM}{\mathcal{M}}   % Calligraphic M
\newcommand{\NN}{\mathcal{N}}   % Calligraphic N
\newcommand{\OO}{\mathcal{O}}   % Calligraphic O
\newcommand{\PP}{\mathcal{P}}   % Calligraphic P
\newcommand{\QQ}{\mathcal{Q}}   % Calligraphic Q
\newcommand{\RR}{\mathcal{R}}   % Calligraphic R
\renewcommand{\SS}{\mathcal{S}} % Calligraphic S
\newcommand{\TT}{\mathcal{T}}   % Calligraphic T
\newcommand{\UU}{\mathcal{U}}   % Calligraphic U
\newcommand{\VV}{\mathcal{V}}   % Calligraphic V
\newcommand{\WW}{\mathcal{W}}   % Calligraphic W
\newcommand{\XX}{\mathcal{X}}   % Calligraphic X
\newcommand{\YY}{\mathcal{Y}}   % Calligraphic Y
\newcommand{\ZZ}{\mathcal{Z}}   % Calligraphic Z

% Tikz
\tikzset{
  arrow symbol/.style={"#1" description, allow upside down, auto=false, draw=none, sloped},
  subset/.style={arrow symbol={\subset}},
  cong/.style={arrow symbol={\cong}}
}

% Specific commands

% Logic
\newcommand{\notb}[1]{\br{\neg #1}}               % Negation
\newcommand{\orb}[2]{\br{#1 \lor #2}}             % Disjunction
\newcommand{\andb}[2]{\br{#1 \land #2}}           % Conjunction
\newcommand{\impb}[2]{\br{#1 \rightarrow #2}}     % Implication
\newcommand{\iffb}[2]{\br{#1 \leftrightarrow #2}} % Biconditional
\newcommand{\fab}[1]{\br{\forall #1}}             % Universal quantifier
\newcommand{\teb}[1]{\br{\exists #1}}             % Existential quantifier
\newcommand{\eqb}[2]{\br{#1 = #2}}                % Equal to
\newcommand{\ltb}[2]{\br{#1 < #2}}                % Less than
\newcommand{\leb}[2]{\br{#1 \le #2}}              % Less than or equal to
\newcommand{\neb}[2]{\br{#1 \ne #2}}              % Not equal to
\newcommand{\inb}[2]{\br{#1 \in #2}}              % An element of
\newcommand{\nib}[2]{\br{#1 \notin #2}}           % Not an element of
\newcommand{\subb}[2]{\br{#1 \subseteq #2}}       % Subset of

% Number theory
\newcommand{\jacobi}[2]{\br{\tfrac{#1}{#2}}}        % Jacobi symbol
\newcommand{\Unit}[1]{\br{\dfrac{\Z}{#1\Z}}^\times} % Unit group vertical
\newcommand{\unit}[1]{\br{\Z / #1\Z}^\times}        % Unit group horizontal

% Front matter

% Fancy header
\pagestyle{fancy}
\lhead{\module}
\rhead{\nouppercase{\leftmark}}

% Make title
\title{\module}
\author{Lectured by \lecturer \\ Typed by \typist}
\date{\term}